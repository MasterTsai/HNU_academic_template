% !TEX encoding = UTF-8 Unicode
\documentclass[10pt,aspectratio=169]{beamer} % 169 为4:3比例
\setbeamercovered{transparent=20}
\usetheme[
%  showheader,
%  red,
%  purple,
%  gray,
%  lightBlue,
  colorblocks,
%  noframetitlerule,
]{scnu}

\usepackage[T1]{fontenc}
\usepackage[utf8]{inputenc}
\usepackage{lipsum}
\providecommand{\lishu}{\CJKfamily{song}}
\usepackage{tikz}
\usetikzlibrary{fadings}
%\setbeamertemplate{sections/subsections in toc}[ball]
\usepackage{xeCJK}
\usepackage{listings}
\usepackage{caption}
\usepackage{subcaption}
\usefonttheme{professionalfonts}
\def\mathfamilydefault{\rmdefault}
\usepackage{amsmath}
\usepackage{multirow}
\usepackage{booktabs}
\usepackage{hyperref}
\usepackage{bm}
\setbeamertemplate{section in toc}{\hspace*{1em}\inserttocsectionnumber.~\inserttocsection\par}
\setbeamertemplate{subsection in toc}{\hspace*{2em}\inserttocsectionnumber.\inserttocsubsectionnumber.~\inserttocsubsection\par}
\setbeamerfont{subsection in toc}{size=\large} % 设置目录样式
%---------------------------------------------------------------------
%在章节开始前放映章节目录
\AtBeginSection[]{%
\begin{frame}%
\frametitle{Outline}%
\textbf{\tableofcontents[currentsection]} %
	\end{frame}%
}
%--------------------------------------------------------------------
% 在子章节开始前放映子章节目录
%\AtBeginSubsection[]{%
%	\begin{frame}%
%		\frametitle{Outline}%
%		\textbf{\tableofcontents[currentsection, currentsubsection]} %
%	\end{frame}%
%}
%===================================================================
\title{\href{https://www.econometricsociety.org/publications/econometrica/2021/03/01/quantile-factor-models}{Quantile Factor Models}}
\subtitle{Theory and Application}   % 子标题
\author[Xiao Cai]{蔡骁}
\mail{\href{mailto:iamcaixiao@hnu.edu.cn}{iamcaixiao@hnu.edu.cn}}
\institute[Hunan University]{\normalsize College of Finance and Statistics, \\Hunan University}
\date{\today}
\titlegraphic[width=3cm]{logo_hnu.jpg}{}  % 在首页显示校徽
%===================================================================
\begin{document}

\maketitle	

\begin{frame}
	\frametitle{Outline}
	\textbf{\tableofcontents}
\end{frame}	

%============================FRAME===========================
\section{Basics}
\subsection{Blocks1}
\begin{frame}[c]{Blocks1}
	
The blocks are shown below
\begin{block}{Regular Block}
	Content of a regular block
\end{block}

\begin{exampleblock}{Example Block}
	Content of an example block
\end{exampleblock}

\begin{alertblock}{Alert block}
	Content of an alert block
\end{alertblock}

\end{frame}	
%==================================================
\subsection{Blocks2}
\begin{frame}[c]{Blocks2}
	
The blocks are shown below
\begin{block}{Regular Block}
	Content of a regular block
\end{block}

\begin{exampleblock}{Example Block}
	Content of an example block
\end{exampleblock}

\begin{alertblock}{Alert block}
	Content of an alert block
\end{alertblock}

\end{frame}	


%==================================================
\section{Pause test}
\begin{frame}{Pause test}
\pause
Hello
\pause[3]
暂停
\end{frame}
%=======================================================================
\section{Item}
\begin{frame}{item}
\begin{itemize}[<+->]
	\item 1\\
	\begin{itemize}
		\item 2
	\end{itemize}
\end{itemize}
\begin{quotation}
	123
\end{quotation}
\end{frame}
%=======================================================================
% Thank you page
\beamertemplateshadingbackground{structure.fg!30}{structure.fg} % 两层设置渐变色
\begin{frame}%[plain]
	\vfill
	\centering
	{
		\centering \Huge \color{white}End
	}
	\vfill
\end{frame}
%===============================================================
\end{document}
